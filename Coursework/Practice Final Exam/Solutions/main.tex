
\documentclass{article}
\usepackage[utf8]{inputenc}
\usepackage{amsmath}
\usepackage{booktabs}
\usepackage{color}
\usepackage{lipsum}
\usepackage{amsmath}
\usepackage{amssymb}
\usepackage{fancyhdr}
\usepackage[T1]{fontenc}
\usepackage{fourier}
\usepackage{listings}
\usepackage{graphicx}
\usepackage{physics}
\usepackage{listings}
\usepackage{tikz}
\usepackage{pifont}
\newcommand{\xmark}{\ding{55}}%
% document size parameters
\def\checkmark{\tikz\fill[scale=1.5](0,.35) -- (.25,0) -- (1,.7) -- (.25,.15) -- cycle;} 
\setlength{\oddsidemargin}{0.0in}
\setlength{\evensidemargin}{0.0in}
\setlength{\textheight}{8.4in}
\setlength{\textwidth}{6.5in}
\setlength{\voffset}{-0.40in}
\setlength{\headsep}{26pt}
\setlength{\parindent}{0pt}
\setlength{\parskip}{6pt}
%commands 
\newcommand{\MyTikzmark}[2]{%
     \tikz[overlay,remember picture,baseline] \node [anchor=base] (#1) {$#2$};%
}

\newcommand{\DrawVLine}[3][]{%
  \begin{tikzpicture}[overlay,remember picture]
    \draw[shorten <=0.3ex, #1] (#2.north) -- (#3.south);
  \end{tikzpicture}
}

\newcommand{\DrawHLine}[3][]{%
  \begin{tikzpicture}[overlay,remember picture]
    \draw[shorten <=0.2em, #1] (#2.west) -- (#3.east);
  \end{tikzpicture}
}

% header information
\pagestyle{fancyplain}
\lhead{\large{{\bf  Numerical Analysis} }}
\rhead{\large{{\bf Solutions}}}
\author{Ramiro Gonzalez, Xavier Villa} 
\title{Practice Final Exam}
\date{Spring 2018}
\begin{document}
\maketitle


\textbf{Here a list of things a student should be able to do to pass the class of Math 131. Work out the solutions as a preparation for the final exam.}


\begin{enumerate}

% Problem 1
    \item Chapter 1\color{green}\checkmark\color{black}  \color{green}\Huge{100\%}\color{black} \large{}
    \begin{enumerate}
        \item State Taylor’s theorem, explain all notations. Approximate $e^{3x}-1$ about 0 with the second Taylor’s polynomial.\\
        \\
        \color{red}
            \textbf{Verbose Answer:}\\
            The following is necessary in order to apply Taylor's theorem.\\
            \textit{$f \in C^n[a,b], f^{n +1}$ exist on given interval and $x_0 \in [a,b]$ there exists a number
            $\xi(x)$ between $x_0$ and $x$.}\\
            \textbf{Layman Translation}:\\ $C^n$ means Continuous differential nth derivative. $f^{n + 1}$ is the next derivative.\\
            \textit{Taylor's theorem states that if there is a function $f(x)$ that is continuous differential n times on [a,b] and is continuous on the (n + 1) derivative exist on the given interval and there is an $x_0$ in that interval, then there is a number that we call $\xi(x)$ that exists in between $x_0$ and $x$ }\\\\
            Enumerated Answer:
            \begin{enumerate}
                \item We know that $f(x) = P_n(x) + R_n(x) $
                \item The function $f(x)$ is defined by the \textbf{nth Taylor polynomial} $P_n(x)$ plus the \textbf{remainder term} $R_n(x)$ nth Taylor Polynomial.
                \item $P_n(x)$ \textbf{nth Taylor Polynomial}.\\ \textbf{n} is the degree, for example "second Taylor Polynomial".\\$f^{k}(x_0)$ is the kth derivative evaluated at the point of interest. \\
                $\sum_{k = 0}^{n}$ is summation from k = 0 to n which is the degree.
                $$P_{n}(x) = \sum_{k = 0}^{n} \frac{f^{k}(x_0)}{k!}(x - x_0)^{k} $$
                Since $x_0 = 0$ then called the \textbf{Maclaurin Polynomial}
                $$P_{n}(x) = \sum_{k = 0}^{n} \frac{f^{k}(x_0)}{k!}(x)^{k} $$
                \item $R_n(x)$ is the \textbf{Remainder term} or \textbf{Truncation Error}\\
                $f^{n + 1}$ is the nth + 1 derivative, that is one more degree. \\
                $\xi(x)$ is a number between $x_0$ and $x$. 
                $$R_{n}(x) = \frac{f^{n + 1}(\xi(x)}{(n + 1)!}(x - x_0)^{n + 1}$$
                Now we approximate $e^{3x} - 1$ about x = 0, second taylor
                \begin{enumerate}
                    \item Let $f(x) = e^{3x} - 1$, $x_0 = 0$
                    \item Find n + 1 derivatives $f'(x) = 3e^{3x}, f''(x) = 9e^{3x}, f^3(x) = 27e^{3x}$. $f^{3}$ is for the truncation. 
                    \item plug  into the formula. Lets expand first.
                    $$P_n(x) = f(x_0) + f'(x_0)(x - x_0) + \frac{f''(x_0)(x - x_0)^2}{2}$$
                    \item Plug in
                    $$P_n(x) = 0 + 3(x) + \frac{9}{2}(x)^2$$
                    \item 
                    $$R_{n}(x) = \frac{f^{3}(\xi(x)}{(3)!}(x - x_0)^{3}$$
                    $$R_{n}(x) = \frac{27e^{3\xi(x)}}{(3)!}(x - x_0)^{3}$$
                    \item Therefore. 
                    $$f(x) = 3(x) + \frac{9}{2}(x)^2 + \frac{27e^{3\xi(x)}}{(3)!}(x)^{3}$$
                \end{enumerate}
            \end{enumerate}
        \color{black}
        
        \item Define 3 types of errors that can be used to compare a number and its approximation.\\
        \\
        \color{red}
            Answer:\\
            $p*$ is an approximation to p. \\
            p is the (actual value) or a number.\\
            The following provides how close the approximation is to the number.\\
            \begin{enumerate}
                \item Relative error:  $\abs{\frac{p-p*}{p}}$ if $p \neq 0$
                \item Absolute error:  $\abs{p-p*}$
                \item Actual error:    $p-p*$
        \end{enumerate}
        \color{black}
    \end{enumerate}
        
%Problem 2
    \item Chapter 2 
    \color{green}\checkmark\color{black} \color{green}\Huge{100\%}\color{black} \large{}\\
    Consider the function $f(x) = (x-1)^2-1 \text{ for } x \in [0.5, 3]$.
    \begin{enumerate}
        \item Can you use bisection method to solve $f(x) = 0$ ? Justify your answer, and if so, provide the first 2 steps.\\
        \\
        \color{red}
            Answer:\\
            Bisection Method cannot guarantee an exact solution to the function, but it can be used to \textit{approximate} the solution if $f$ is a continuous function defined on the interval $[a, b]$, with $f(a)$ and $f(b)$ of opposite sign. The Intermediate Value Theorem implies that a number $p$ exists in $(a, b)$ with $f(p) = 0$. If there is more then one root within the interval it will not find both.\\\\
            Enumerate Answer:\\
            To solve $f(x) = 0$ we must show that the Intermediate Value Theorem Applies. 
            \begin{enumerate}
                \item Show $f(x)$ is continuous on the interval [a,b]
                $$f'(x) = 2(x-1)$$ 
                Since $f'(x)$ is a polynomial it is continuous everywhere thus on [.5,3].
                \item Evaluate at endpoints that is $f(a)$ and $f(b)$
                $$f(.5) = -.75 \text{ and } f(3) = 3$$
                \item Show $f(a)\cdot f(b) < 0$ that is they have opposite signs. 
                $$f(.5)\cdot f(3) < 0$$
                Therefore the IVT is met. The bisection method applies since we can guarantee there is at least one root. 
            \end{enumerate}
            First 2 steps:
            \begin{enumerate}
                \item[step 1] Let $a_1 = .5$ and $b_1 = 3$
                $$f(.5) = -.75 \text{ and } f(3) = 3$$
                 $p_1 = \frac{ a + b}{2} = \frac{3.5}{2} = 1.75$ the \textbf{midpoint}\\
                 $$f(p_1) = f(1.75) = -0.4375$$
                 Since $f(p_1)$ has the same sign as $f(.5)$ our we have a new a, named $a_2 = p_1$. Note that $b_2 = b$ stays the same.\\
                 \item[step 2] Let $a_2 = 1.75$ and $b_2 = 3$
                 $$f(1.75) = -.4375 \text{ and } f(3) = 3$$
                 $p_2 = \frac{a + b}{2} = \frac{4.75}{2}  = 2.375$\\
                $$f(p_2) = f(2.375) =  0.890625$$
                 Since $f(p2)$ has same sign as b, therefore we have a new $b_3 = p_2$. We do not continue. \\
            \end{enumerate}
            The goal is for the interval $[a,b]$ to become increasingly small. 
        \color{black}
        \item Rewrite this problem as a fixed-point problem $g(x) = x$, where $g$ is a function to define. What is the theoretical rate of convergence of the fixed-point iteration? Is the fixed-point converging in that case? Justify your answer.\\
        \\
        \color{red}
            Enumerated Answer:\\
            \begin{enumerate}
                \item We are given $f(x) = (x-1)^2 - 1$
                \item We want to find $g(x)$ 
                \item[Option1 ] You can isolate x on $f(x) = (x-1)^2 - 1$
                $$f(x) = x^2 - 2x + 1 - 1 = x^2 - 2x$$
                $$f(x) = (x^2 - 2x)$$
                $$\text{set f(x) = 0 } x^2 - 2x = 0 $$
                $$x^2 - 2x = 0 \rightarrow 2x = x^2 \text{ solve for x} $$
                $$g(x) = x = \frac{x^2}{2} \text{ this is our } g(x)$$
                \item[Option 2]You can add zero, that is $f(x) = (x -1)^2 - 1 + (x - x)$ and move one x to the other side. 
                $$g(x) = x = (x-1)^2 - 1 + x \text{ this is our g(x)}$$
            \end{enumerate}
            The theoretical rate of convergence is $\mathcal{O}(k^n)$ where n is the number of iterations and k is the bound. \\\\
            The fixed point converges if it is bounded, that is $0 \leq k \leq 1$ the smaller the k the faster it converges. \\
            We want to show there exist a k such that $|g'(x)| \leq 1$ . Chose either g(x)
            \begin{enumerate}
                \item Show g(x) maps onto itself
                $$g(x) = \frac{x^2}{2}$$
                $g(.5) = .125$,  Failed, convergence can not confirmed. 
                \item or $$g(x) = \frac{x^2}{2}$$
            $$|g'(x)| = |x|$$
            if $g'(x = .5) = .5 < 1$ and $g'(x = 3) = 3 > 1$
            Therefore fixed point is not converging. The bound not found. 
            \end{enumerate}
            
        \color{black}
        \item Using Newton’s method, provide $x_1$ for $x_0 = 1.5$. Show steps. What is the order of convergence of the sequence of error? Explain the limitations and advantages of this method compared to other numerical methods for root-finding problems.\\
        \\
        \color{red}
            Answer:\\
           \begin{enumerate}
                \item Our initial guess is $x_0 = 1.5$, $$f(x) = (x -1)^2 - 1$$
                $$f'(x) = 2(x - 1)$$
                \item Before you begin, make sure $f'(x_0) \neq = 0$\\
                $f(1.5) = -.75$, $f'(1.5) = 2(.5) = 1$
                \item Newton's method is as follows. 
                $$x_i = x_{i - 1} - \frac{f(x_{i - 1})}{f'(x_{i - 1})}$$
                \item We want to find $x_1$ thus $i = 1$
                $$x_1 = x_{0} - \frac{f(x_{0})}{f'(x_{0})}$$
                \item We evaluate
                $$x_1 = 1.5 - \frac{-.75}{1} = 2.25$$
                The order of convergence is Quadratic for Newton's method.
            \end{enumerate}
            \begin{tabular}{c|c}
                 Advantages &  Disadvantages \\
                 \hline
            Accurate approximations with very little iterations.   &
            Our initial approximation $x_0$ 
            has to be close to root\\
            \hline
            Whether it converges or not is clear & We must find derivative
            \end{tabular}
        \color{black}
     \end{enumerate}
     
     % Problem 3
    \item Chapter 3 \color{green}\checkmark\color{black}
    \begin{enumerate}
        \item Given $x_0 = 1, x_1 = 1.5, x_2 = 2, \text{ and } y_0 = 3.4, y_1 = 1.2, y_2 = -3, \text{ provide the Lagrange interpolant passing}$\\
        $\text{ through }  (x_k , y_k ) \text{ for } k = 0, 1, 2.$ Explain all your notations and provide steps\\
        \\
        \color{red}
            Answer:\\
            The Lagrange Polynomial is an interpolation method and is as follows. 
            $$P_n(x) = \sum_{k = 0}^{n} f(x_k)\mathcal{L}_{n,k}(x)$$
            $P_n(x)$ is called the \textbf{nth Lagrange Interpolating Polynomial}\\
            n is called the \textbf{nth (degree)}\\
            $x_0, x_1,...x_k$ are the \textbf{ nodes}\\
            $f(x_0) = y_0....f(x_k) = y_k$ are \textbf{ nodes evaluated by the function. }\\
            The following shows the $\mathcal{L}_{n,k}$ \textbf{Called the Lagrange Basis Polynomial}
            $$\text{where }  \mathcal{L}_{n,k}(x) = \frac{\Pi_{j=0, j\neq k}^{n}(x-x_j)}{\Pi_{j=0, j\neq k}^{n}(x_k-x_j)}$$
            \\
            \begin{enumerate}
                \item First we want to evaluate the Lagrange Basis Polynomials where our degree is 2. And our $k$ will iterate $0,1,2$
                $$ \mathcal{L}L_{2,0}(x) = \frac{(x-x_1)(x-x_2)}{(x_0-x_1)(x_0-x_2)} = 2x^2 -7x +6 $$
                 $$ \mathcal{L}_{2,1}(x) = \frac{(x-x_0)(x-x_2)}{(x_1-x_0)(x_1-x_2)} = -4x^2 + 12x -8 $$
                  $$ \mathcal{L}_{2,2}(x) = \frac{(x-x_0)(x-x_1)}{(x_2-x_0)(x_2-x_1)} = 2x^2 -5x +3 $$
                 \item The goal is to evaluate as the following form 
                 $$ P(x) = y_0\mathcal{L}_{2,0}(x) + y_1\mathcal{L}_{2,1}(x) + y_2\mathcal{L}_{2,2}(x) $$
                 \item Evaluate (plug in)
                   $$ P(x) = 3.4(2x^2 -7x +6) + 1.2(-4x^2 + 12x -8) - 3(2x^2 -5x +3) $$
                  $$ P(x) = (6.8x^2 -23.8x +20.4) + (-4.8x^2 + 14.4x -9.6) + (-6x^2 +15x - 9) $$
                   $$ P(x) = (6.8x^2 -4.8x^2 -6x^2) + 
                   (-23.8x + 14.4x  +15x) + (20.4 -9.6 - 9) $$
                   $$ P(x) = -4x^2 + 5.6x +  1.799$$
            \end{enumerate}
            
        \color{black}
        
        \item Given $x_0 = 1, x_1 = 1.5, x_2 = 2, \text{ and } y_0 = 3.4, y_1 = 1.2, y_2 = -3$, provide all the coefficients needed in Newton’s divided differences, and give the expression of the created interpolant. Explain all your notations and provide steps.\\
        \\
        \color{red}
            \begin{enumerate}
                \item First we define the expression for Newton Divided Difference. 
                $$ P(x) = f(x_0) + \sum_{k=1}^{n}f[x_0, \dots, x_k](x-x_0) \dots(x-x_{k-1}) $$
                \item Consider the degree n = 2, we need the following coefficients. 
                \[
            \begin{array}{cccccc}
            f[x_0] \\
                &       f[x_0, x_1] \\
            f[x_1] &             & f[x_0, x_1, x_2]\\
                &       f[x_1, x_2] \\
            f[x_2]
            \end{array}
            \]
            \item Memorize the following. 
            $$ f[x_0], f[x_1], \text{ and } f[x_2] \text{ are given.} $$
            $$ f[x_0, x_1] = \frac{f[x_1]-f[x_0]}{x_1-x_0} = -\frac{22}{5} $$
            $$ f[x_1, x_2] = \frac{f[x_2] - f[x_1]}{x_2 - x_1} = \frac{-42}{5} $$
            $$ f[x_0, x_1, x_2] = \frac{f[x_1, x_2] - f[x_0, x_1]}{x_2 - x_0} = -4 $$
            \item We want this form 
            $$ P(x) = f[x_0] + f[x_0, x_1](x-x_0) + f[x_0, x_1, x_2](x-x_0)(x-x_1) $$
            \item Plug $$ P(x) = 3.4 + \frac{-22}{5}(x - 1) - 4(x-1)(x - 1.5) $$
            $$ P(x) = 3.4 + \frac{-22}{5}(x - 1) - 4[x^2 -2.5x + 1.5]) $$
            $$ P(x) = 3.4 + \frac{-22}{5}x - \frac{-22}{5} - 4x^2 + 4*2.5x - 4*1.5)$$
            $$ P(x) = - 4x^2 + \frac{28}{5} \frac{9}{5}$$

            \end{enumerate}
            
            
        \color{black}
        
        \item Explain in one sentence what is a cubic spline interpolant.\\
        \\
        \color{red}
            Answer:\\
            The Cubic Spline Interpolant is a \textbf{piece-wise}, \textbf{3rd order polynomial} construction that passes through a set of control points. (interpolates points using a polynomial of degree 3)
        \color{black}
    \end{enumerate}
    
    % Problem 4
    \item Chapter 4 \color{green}\checkmark\color{black}\\
    Consider $x_0 = 0, x_1 = 0.5, x_2 = 1, x_3 = 1.5, f(x_0) = 3.4, f(x_1) = 1.2, f(x_2) = -3, f(x_3) = -1$, and the integral $I = \int_0^{1.5}f(x)dx$.
    \begin{enumerate}
        \item Use the 3-point midpoint rule to approximate $f'(1)$.\\
        \\
        \color{red}
            Answer:\\
            \begin{enumerate}
                \item Define the 3-point midpoint rule. 
                $$ f'(x)\approx \frac{1}{2h}(f(x +h)-f(x -h)) $$
                \item find h, $h = x_1 - x_0, x_2 - x_1 = .5$
                Where $h$ is the distance between the given $x$ values
                \item x = 1, we want to find $f'(x)$
                $$ f'(1)\approx \frac{1}{2(0.5)}(f(1 +0.5)-f(1 -0.5)) $$
                $$ f'(1)\approx f(1.5)-f(0.5) = f(x_3) - f(x_1) = -1 - 1.2 = -2.2$$
            \end{enumerate}
        \color{black}
        
        \item Provide the General formula for the Composite Trapezoid rule. Explain all your notations. Apply Composite Trapezoid rule to approximate $I$.\\
        \\
        \color{red}
            Enumerated Answer:\\
            \begin{enumerate}
                \item Original Definition.\\
                a is the lower bound and b is the upper bound.\\ $[a,b]$ is an interval.\\
                h is defined as $h = \frac{b - a}{N} $
                
                $$\int_{a}^{b}f(x)dx = \frac{h}{2}\Big[f(a) + 2 \sum_{j = 1}^{n - 1}f(x_j) + f(b)\Big]$$
                \item Move around
                $$\int_{a}^{b}f(x)dx = \frac{h}{2}\Big(f(a) + f(b)\Big) + \frac{h}{2}\Big[ 2\sum_{j = 1}^{n - 1}f(x_j)\Big]$$
                $$\int_{a}^{b}f(x)dx = \frac{h}{2}\Big(f(a) + f(b)\Big) + h\Big[\sum_{j = 1}^{n - 1}f(x_j)\Big]$$
                
                \item Now we solve for the following. $h = \frac{1.5 - 0}{3} = \frac{1}{2}$
                \item Evaluate
                 $$\int_{0}^{1.5}f(x)dx = \frac{.5}{2}\Big(3.4 - 1\Big) + (.5)\Big[\sum_{j = 1}^{n - 1}f(x_j)\Big]$$
                 $$\int_{0}^{1.5}f(x)dx = \frac{.5}{2}\Big(3.4 - 1\Big) + (.5)\Big[(1.2 - 3)\Big]$$
                 $$\int_{0}^{1.5}f(x)dx = \text{ Left to the Reader, \textbf{should be -0.3}}$$
                 
            \end{enumerate}
            Other Answer:\\
            The Trapezoidal rule relies on this basic formula:
            $\int_{b}^{a}f(x)dx \approx \frac{h}{2}(f(x_a)+(x_b))$
            The composite Trapezoidal rule does this by breaking up $a$ and $b$ into smaller segments and attractively adds up each result to get a closer approximation for the integral across the interval.\\
            \\
            Thus, the general formula for Composite Trapezoidal Rule is:
            $$ I = I + \frac{h}{2}(f(x_n) + f(x_{n+1})) $$
            $$\text{for }n = 0\rightarrow N-1$$
            \item Where $I = 0$ initially, N is the number of $x$ values being evaluated across our interval, $h$ (timestep) is the distance between the $x$ values, $n$ is the $n^{\text{th}}$ iteration and $x_n$ and $x_{n+1}$ are the $x$ values being evaluated in each $n^{th}$ iteration.
        \color{black}
        
        \item State precisely $3$ differences between Composite Trapezoid rule and Composite Simpson’s rule.\\
        \\
        \color{red}
            Answer:\\
            \begin{enumerate}
            \item The error term in Simpson’s rule involves the fourth derivative of $f$, so it gives exact results when applied to any polynomial of degree three or less. The Trapezoidal rule however gives zero error for polynomials of degree one our less.
            \item Simpson's rule requires 3 $x$ values for each iteration, and the Trapezoidal rule requires only 2 per iteration.
            \item The order of convergence for Trapezoidal rule is $O(h^2)$ and order of convergence for Simpson's is $O(h^4)$
            \item  Trapezoid consists of the summation of trapezoid, that is there is a rectangle and a triangle, a slanted line. Simpson’s rule uses a rectangle and a parabola, making it slightly more accurate.
            \end{enumerate}
        \color{black}
        
    \end{enumerate}
    
    % Problem 5
    \item Chapter 5 \color{green}\checkmark\color{black}\\
    Consider the IVP $y'(t) = -4y,\: 1 \leq t \leq 4, y(1) = 4$.
    \begin{enumerate}
        \item Show that this problem is well-posed, THEN provide the exact solution.\\
        \\
        \color{red}
            Answer:\\
            \begin{enumerate} 
                \item To show the problem is well-posed, we must show the Lipschitz condition is met.\\
                $ D = \{(t,y)|a \leq t \leq b \text{ and } -\infty < y < \infty\}$
                \\
                \item Make sure $f(t,y)$ is continuous, it is a polynomial therefore it is. 
                \item Let $f(t,y) = -4y$
                $$\Big|\frac{f(t,y_2) - f(t,y_1)}{y_2 - y_1}\Big| \leq \frac{\partial}{\partial y}(-4y)$$
                $$\|f(t,y_2) - f(t,y_1)| \leq \frac{\partial}{\partial y}|-4||y_2 - y_1|$$
                \item We have found a constant L = 4 Therefore the Lipschitz condition is met. $f(t,y)$ is continuous and meets the Lipschitz condition therefore it is well posed. 
            \end{enumerate}
        Next we find the exact solution. 
        $$\frac{dy}{dt} = -4y \text{ separable differentiable}$$
        $$\frac{1}{-4y}dy = dt \rightarrow -\frac{\ln(y)}{4} = t + C$$
        $$y = e^{-4t}e^{-4c}$$
        $$y = e^{-4t}C_1$$
        $$C = 4e^{4}$$
        Solution 
        $$y(t) = 4e^{-4t + 4}$$
        \color{black}
       
        
        \item Apply Euler’s method to approximate the solution of the IVP. considering a time-step $\Delta t = 0.5$. Explain all notation, give only the general and simplified expression for some index j and comment on the needed initial value(s)..\\
        \\
        \color{red}
            Answer:\\
            \begin{enumerate}
            \item Euler's method is defined by the following algorithm:
            $$y_{i+1} = y_{i} + hf(t_{i},y_{i}) $$
            \item For $i = 0\rightarrow (N-1)$ where $N$ is the number of points being evaluated, $h$ (timestep, $\Delta t$) is the distance between these points, and $f(t_i, y_i)$ is the function given evaluated at $t_i$, and $y_i$, the values being evaluated for in each iteration.
            \item Our values of $t_i$ are calculated by $t_{i+1} = t_i + h$ or equivalently, $t_{i+1} = t_i + \Delta t$
            \item With time-step $\Delta t = h = 0.5,$ we have the following for out values of $t$:
            $$ t_0 = 1,\;  t_1 = 1.5,\;  t_2 = 2.0,\;  t_3 = 2.5,\;  t_4 = 3.0,\;  t_5 = 3.5,\;  t_6 = 4.0 $$
            \item Since we are evaluating 7 points from 0 to $N-1$, our $i$ will take on values $0, 1, \dots, 6 $
            \item Begin solving, with $y_0 = \alpha$ defined as given by: $y(1) = 4 = \alpha$
            $$ y_1 = y_0 + hf(t_0, y_0) $$
            $$ y_2 = y_1 + hf(t_1, y_1) $$
            $$ \vdots $$
            $$ y_6 = y_5 + hf(t_5, y_5) $$
            $$ y_1 = 4 + \frac{1}{2}4(-4) = -4$$
            $$ y_2 = 12 + \frac{1}{2}(-4)(-4) = 4 $$
            \item Here we notice this pattern is recursive.
            $$ y_3 = -4 $$
            $$ y_4 = 4 $$
            $$ y_5 = -4 $$
            $$ y_6 = 4 $$
            Thus we have everything for our solution.\\
            Euler's method returns:\\
            \\
            \begin{tabular}{c|c}
                 t_i &  y_i \\
                 \hline
            t_0 = 1 &  y_0 = 4\\
            t_1 = 1.5 &  y_1 = -4\\
            t_2 = 2 &  y_2 = 4\\
            t_3 = 2.5 &  y_3 = -4\\
            t_4 = 3 &  y_4 = 4\\
            t_5 = 3.5 &  y_5 = -4\\
            t_6 = 4 &  y_6 = 4\\
            \end{tabular}
            \end{enumerate}
        \color{black}
        
        \item Provide a stable implicit 3-step method to approximate the solution of the IVP. Explain all notations, and comment on the needed initial value(s).\\
        \\
        \color{red}
            Answer:\\
            Let's make up a method that is 3-step implicit, and test its stability.\\
            $$ W_{j+1} = 2W_{j+1} - W_{j} + W_{j-1} - W_{j-2} + f(t, W_j) $$
            The first step is to find the characteristic polynomial by setting $W_{j+1} = \lambda^m$ Where $m$ is the number of steps (3).
            $$ W_{j+1} = \lambda^3 $$
            $$ \lambda^3 = 2\lambda^3 - \lambda^2 + \lambda -1 $$
            $$ \lambda^3 - \lambda^2 + \lambda -1 = 0 $$
            Now we solve for the roots to evalutate the stability.
            $$ \lambda^2(\lambda -1)(\lambda -1) = 0 $$
            $$ (\lambda -1)(\lambda^2 + 1) = 0 $$
            $\therefore$ The roots of the characteristic polynomial are $\lambda = \{i, -i, -1\}$
            This satisfies the root condition, since none of the absolute value of the roots are greater than 1. This made-up method is strong stable because are only one roots that are 1.
        \color{black}
        
    \end{enumerate}
    
    % Problem 6
    \item Chapter 6\color{green}\checkmark\color{black}\\
    Consider the system $Ax = b$ with $A = \begin{pmatrix} 1 & 2 & 3\\ 2 & 1 & 2\\ 3 & 2 & 1 \end{pmatrix}$ and $B = \begin{pmatrix} 1\\ 2\\ 3 \end{pmatrix}$
    \begin{enumerate}
        \item Compute $det(A)$ and conclude some properties about the system.\\
        \color{red}
            Answer:\\
            $3 \cross 3$ Matrix.
            
            \begin{enumerate}
                \item Finding det(A)\\

\begin{bmatrix}
    \MyTikzmark{leftA}{1} &  \MyTikzmark{topB}{2} &  \MyTikzmark{rightA}{3} \\
     2 & 1 & 2 \\
   \MyTikzmark{bottomA}{3} & 2 & \MyTikzmark{bottomB}{1} 
\end{bmatrix}
\DrawVLine[red, thick, opacity=0.5]{leftA}{bottomA}
\DrawHLine[blue, thick, opacity=0.5]{leftA}{rightA} = 1(1 - 4) = -3

\begin{bmatrix}
    \MyTikzmark{leftA}{1} &  \MyTikzmark{topB}{2} &  \MyTikzmark{rightA}{3} \\
     2 & 1 & 2 \\
   \MyTikzmark{bottomA}{3} &  \MyTikzmark{bottomB}{2} & \MyTikzmark{bottomA}{1} 
\end{bmatrix}
\DrawVLine[red, thick, opacity=0.5]{topB}{bottomB}
\DrawHLine[blue, thick, opacity=0.5]{leftA}{rightA} = 2(2 - 6) = -8



\begin{bmatrix}
    \MyTikzmark{leftA}{1} &  \MyTikzmark{topB}{2} &  \MyTikzmark{rightA}{3} \\
     2 & 1 & 2 \\
   \MyTikzmark{bottomA}{3} &  \MyTikzmark{bottomB}{2} & \MyTikzmark{bottomA}{1} 
\end{bmatrix}
\DrawVLine[red, thick, opacity=0.5]{rightA}{bottomA}
\DrawHLine[blue, thick, opacity=0.5]{leftA}{rightA} = 3(4 - 3) =3
\item det(A) = -3 + 8 + 3 = 8
\begin{enumerate}
    \item Since det(A) $\neq 0$ the matrix is invertible.
    \item the rows and columns are linearly independent. 
    \item It is also symmetric and singular. 
\end{enumerate}
            \end{enumerate}
        \color{black}
        
        \item Perform Gaussian Elimination to reduce the system. Show steps.\\
        \\
        \color{red}
            Answer:\\
            \begin{enumerate}
                \item Put A and B in Augmented form \\ \textbf{Left To The Reader}
                $$\begin{pmatrix} 1 & 2 & 3 &\vline & 1\\ 2 & 1 & 2 &\vline & 2 \\ 3 & 2 & 1  & \vline  & 3 \end{pmatrix}$$
                
                $2r_1 - r_2 \rightarrow r_2 $
                $$\begin{pmatrix} 1 & 2 & 3 &\vline & 1\\
                0 & 3 & 4 &\vline & 0 \\ 
                3 & 2 & 1  & \vline  & 3 \end{pmatrix}$$
                
               $3r_1 - r_3 \rightarrow r_3 $
                $$\begin{pmatrix} 1 & 2 & 3 &\vline & 1\\
                0 & 3 & 4 &\vline & 0 \\ 
                0 & 4 & 8  & \vline  & 0 \end{pmatrix}$$
                
                $\frac{4}{3}r_2 - r_3 \rightarrow r_3 $
                $$\begin{pmatrix} 1 & 2 & 3 &\vline & 1\\
                0 & 3 & 4 &\vline & 0 \\ 
                0 & 0 & \frac{-8}{3}  & \vline  & \frac{4}{3} \end{pmatrix}$$
                
                $\frac{-3}{8}r_3 \rightarrow r_3 $
                 $$\begin{pmatrix} 1 & 2 & 3 &\vline & 1\\
                0 & 3 & 4 &\vline & 0 \\ 
                0 & 0 & 1  & \vline  & \frac{-1}{2} \end{pmatrix}$$
                
                $4r_3 - r_2 \rightarrow r_2$
                 $$\begin{pmatrix} 1 & 2 & 3 &\vline & 1\\
                0 & 3 & 0 &\vline & -2 \\ 
                0 & 0 & 1  & \vline  & \frac{-1}{2} \end{pmatrix}$$
                
                $3r_3 - r_1 \rightarrow r_1$
                 $$\begin{pmatrix} 1 & 2 & 0 &\vline & \frac{-5}{2}\\
                0 & 3 & 0 &\vline & -2 \\ 
                0 & 0 & 1  & \vline  & \frac{-1}{2} \end{pmatrix}$$
                $r_2/3 \rightarrow r_2$
                 $$\begin{pmatrix} 1 & 2 & 0 &\vline & \frac{-5}{2}\\
                0 & 1 & 0 &\vline & -2/3 \\ 
                0 & 0 & 1  & \vline  & \frac{-1}{2} \end{pmatrix}$$
                $2r_2 - r_1\rightarrow r_1$
                 $$\begin{pmatrix} 1 & 0 & 0 &\vline & \frac{7}{6}\\
                0 & 1 & 0 &\vline & -2/3 \\ 
                0 & 0 & 1  & \vline  & \frac{-1}{2} \end{pmatrix}$$
            \end{enumerate}
            
        \color{black}
        
        \item Explain what is the LU factorization.\\
        \\
        \color{red}
            Answer:\\ L stands for \textbf{Lower triangular matrix} and U stands for \textbf{Upper triangular matrix} .\\
            LU is a modified form of the Gaussian elimination, The reason for LU factorization is to not have to interchange rows. 
            
        \color{black}
        
    \end{enumerate}
    
    % Problem 7
    \item Chapter 7\color{green}\checkmark\color{black}\\
    Consider the system $Ax = b$ with $A = \begin{pmatrix} 1 & 2 & 0\\ 2 & 1 & 2\\ 0 & 2 & 1 \end{pmatrix}$ and $B = \begin{pmatrix} 1\\ 2\\ 3 \end{pmatrix}$
    \begin{enumerate}
        \item Compute the $l_2$ norm of A.\\
        \\
        \color{red}
            Answer:\\
            Because this matrix is symmetric, $A = A^T$ we can take a shortcut.\\
            We must first find the Eigenvalues. Then we take the largest Eigenvalue, and multiply this value by the largest value in A.
            $$ A = \begin{pmatrix} 1 & 2 & 0\\ 2 & 1 & 2\\ 0 & 2 & 1 \end{pmatrix} $$
            $$ det(A - \lambda I)=0 $$
            $$ A-\lambda I = \begin{pmatrix} 1-\lambda & 2 & 0\\ 2 & 1-\lambda & 2\\ 0 & 2 & 1-\lambda \end{pmatrix} $$
            $$ det(A - \lambda I)=\lambda^3 -3\lambda^2 -5\lambda +7 = 0 $$
            $$ \lambda = {1,\:1+2\sqrt{2},\:1-2\sqrt{2}} $$
            Since $\lambda = 1+2\sqrt{2}$ is the largest Eigenvalue, we can simply multiply that by the largest value of A.
            The largest value in A is 2, so $2(max(\lambda)) = 2(1+2\sqrt{2}) = 2 + 4\sqrt{2}$
        \color{black}
        
        \item Provide the algorithm for Jacobi iteration, and provide all the needed matrices for the given system. Under which condition is this algorithm converging?\\
        \\
        \color{red}
            Answer:\\
            Enumerated Answer:
            \begin{enumerate}
                \item Given \textbf{A} and \textbf{b}. Inital approximation as $x_1^0,x_2^0.....x_n^0$
                \item Make a system of equations, solve for $x_1^1,x_2^1...x_n^1$
                \item Use the initial approximation. Then $x_1^2,x_2^2,,,x_n^2$ is solved
                \item use the second intial approximation to approximate $x_1^3,x_2^3...x_n^3$
                \item continue
                \item It converges when the spectral radius $\rho(M^-1\cdot N) < 1$
            \end{enumerate}
            Also an Answer:\\
            For this Method, we need the given matrix A, the matrices: U, L, and D, the vector $b$, and we will approximate $x$ given an initial guess we call $x_0$
            Recall that $Ax = b$\\
            \\
            $A= \begin{pmatrix} 1 & 2 & 0\\ 2 & 1 & 2\\ 0 & 2 & 1 \end{pmatrix}$\\
            \\
            To get this problem in the form $ A = M - N  $, we will need $U,\:L,\:D$ such that $A = U + L + D$\\
            and U is the upper right portion of A (excluding the diagonal), L is the lower left portion of A (excluding the diagonal), and D is the diagonal of A.
            $ A = M - N $\\
            \\
            $ \begin{pmatrix} 1 & 2 & 0\\ 2 & 1 & 2\\ 0 & 2 & 1 \end{pmatrix} = U + L + D =\begin{pmatrix} 0 & 2 & 0\\ 0 & 0 & 2\\ 0 & 0 & 0 \end{pmatrix} +\begin{pmatrix} 0 & 0 & 0\\ 2 & 0 & 0\\ 0 & 2 & 0 \end{pmatrix} + \begin{pmatrix} 1 & 0 & 0\\ 0 & 1 & 0\\ 0 & 0 & 1 \end{pmatrix} $\\
            \\
            $ A = M - N $\\
            $ \text{let }M = D $\\
            $ \text{let }N = (-U-L) $\\
            $ \text{Such that }A = D - (-U-L) $\\
            $ \text{Since }Ax = b, \text{ We know that } Ax = Dx - (-U-L)x = b $\\
            $ \text{ Rearranging, we have } Dx = b + (-U-L)x $\\
            $ \text{Getting $x$ by itself requires multiplying both sides by the inverse of D.} $\\
            $ x = D^{-1}b + D^{-1}(-U-L)x $\\
            $ \text{We know that our diagonal D is the identity matrix: }\begin{pmatrix} 1 & 0 & 0\\ 0 & 1 & 0\\ 0 & 0 & 1 \end{pmatrix} $\\
            \\
            $\rho$ is the spectral radius. \\
            Leaving us with the Jacobi Algorithm:
            $$x = b + (-U-L)x  $$   
            This algorithm converges if and only if: \\
            $$\rho(M^{-1}\cdot N)<1 \text{ Where } M^{-1} = D^{-1}, \text{ and } N = (-U-L) $$
            Note: Here, $\rho$ denotes the maximum eigenvalue of the resulting Matrix in parenthesis.\\
            Since our $\rho = 1+2\sqrt2 \: >\: 1$, the Jacobi iteration solver is not converging.
            
        \color{black}
        
    \end{enumerate}    

\end{enumerate}
\end{document}