\documentclass{article}
\usepackage[utf8]{inputenc}
\usepackage{amsmath}
\usepackage{booktabs}
\usepackage{color}
\usepackage{lipsum}
\usepackage{amsmath}
\usepackage{amssymb}
\usepackage{fancyhdr}
\usepackage[T1]{fontenc}
\usepackage{fourier}
\usepackage{listings}
\usepackage{graphicx}
\usepackage{physics}
\usepackage{listings}

% document size parameters

\setlength{\oddsidemargin}{0.0in}
\setlength{\evensidemargin}{0.0in}
\setlength{\textheight}{8.4in}
\setlength{\textwidth}{6.5in}
\setlength{\voffset}{-0.40in}
\setlength{\headsep}{26pt}
\setlength{\parindent}{0pt}
\setlength{\parskip}{6pt}

% header information
\pagestyle{fancyplain}
\lhead{\large{{\bf  Numerical Analysis} }}
\rhead{\large{{\bf Solutions}}}
\title{Practice Midterm 1}
\author{Ramiro Gonzalez }
\date{}
\begin{document}
\maketitle
\begin{enumerate}

% PROBLEM 1
\item Let $f(x) = -3x^3 + 4x - 2$
\begin{enumerate}
    \item Use Newton�s method to find $x_1$ if $x_0 = 1$.\\
    Consider that $$x_n = x_{n - 1} - \frac{fx_{n-1}}{f'(x_{n-1})}$$
    to find $x_1$
    $$x_1 = x_0 - \frac{f(x_0)}{f'(x_0)}$$
    We know that $f(1) = -1$, $f'(x) = -9x^2 + 4$, $f'(1) = -5$ therefore
    $$x_1 = 1 + \frac{-1}{5} = \frac{4}{5} = .8$$
    \item Use the Secant method to find $x_2$ if $x_0= 2$ and $x_1= 1$.\\
    Consider that $$x_n = x_{n-1} - \frac{f(x_{n-1})(x_{n-1} - x_{n - 2})}{f(x_{n-1}) - f(x_{n-2})}$$
    $$x_{2} = x_{1} - \frac{f(x_{1})(x_{1} - x_{0})}{f(x_{1}) - f(x_0)}$$
    given that $x_0 = 2, x_1 = 1$, $f(1) = -1, f(2) = -18 $
    $$x_{2} = 1 - \frac{(-1)(1 - 2)}{-1 + 18} = \frac{16}{17} \approx{0.9412}$$
\end{enumerate}

% PROBLEM 2
\item Let $g(x) = \sqrt{x + 6}$
\begin{enumerate}
    \item Show that $c = 3$ is a fixed point of the function $g(x)$\\
    We know that $g(c) = 3$
    $$g(c) = c = \sqrt{c + 6}$$
    $$c^2 - c - 6 = 0$$
    $$(c - 3)(c +2) = 0$$
    The graph $g(x)$ has a fixed point at $c = -2$ and $c = 3$\\
    \\
    Optionally, you may show $g(c) = c$ to be true: $$ g(c) = c$$ $$g(3) = \sqrt{3+6} = \sqrt{9} = 3 $$
    \item Find a bound for $|g'(x)|$ for $x \in [0,4]$\\
    We know that there $c = 3$ is a fixed point, $c \in [0,4]$
    $$g'(x) = \frac{1}{2(\sqrt{x+6})}$$
    $$|g'(x)| = \frac{1}{2}\abs{\frac{1}{\sqrt{x + 6}}}$$
    $$g'(0) = \frac{1}{2\sqrt{6}}, g'(4) = \frac{1}{2\sqrt{10}}$$
    $$\frac{1}{2\sqrt{10}} \leq g'(x) \leq \frac{1}{2\sqrt{6}}$$

    \item Is the fixed-point method converging to the fixed point $c = 3$ in the interval $[0,4]$? If so, what is the rate of convergence?\\
    since $|g'(c)| = \frac{1}{6} < 1$ we know it converges. Since  $|g'(c)| \neq 0$ the rate of convergence is linear. \\
    \color{red} The following shows it :  \color{black}
    Using Taylors Theorem
    $$g(x) = \sqrt{x + 6}, g'(x) = \frac{1}{2\sqrt{x + 6}}, g''(x) = \frac{-1}{4(x + 6)^{\frac{3}{2}}}$$
    $$g(x) = \sqrt{c + 6} + \frac{1}{2\sqrt{c + 6}}(x - c) + (\frac{-1}{4(\xi + 6)^{\frac{3}{2}}})(x - c)^2$$
    We know that $\xi : [x, c]$ , $x_{n +1} = g(x_n)$, we know that 
    $g(c) = c, \sqrt{c + 6} = c$
    $$x_{n + 1} = c + \frac{1}{2\sqrt{c + 6}}(x_{n} - c) + (\frac{-1}{4(\xi + 6)^{\frac{3}{2}}})(x_{n} - c)^2 $$
    $$x_{n + 1} - c = \frac{1}{2\sqrt{c + 6}}(x_{n} - c) + (\frac{-1}{4(\xi + 6)^{\frac{3}{2}}})(x_{n} - c)^2 $$
    $$\frac{{x_{n + 1} - c}}{x_{n} - c} =  \frac{1}{2\sqrt{c + 6}} + (\frac{-1}{4(\xi + 6)^{\frac{3}{2}}})(x_{n} - c) $$
    $$\lim_{n \to \infty}\abs{\frac{{x_{n + 1} - c}}{x_{n} - c}} =  \frac{1}{2\sqrt{c + 6}} + (\frac{-1}{4(\xi + 6)^{\frac{3}{2}}})(x_{n} - c) $$
    $$\text{consider that} \lim_{n \to \infty}(\frac{-1}{4(\xi + 6)^{\frac{3}{2}}})(x_{n} - c) = 0, (\lim_{n \to \infty} (x_n) = c)$$
    $$\lim_{n \to \infty}\abs{\frac{{x_{n + 1} - c}}{x_{n} - c}} =  \frac{1}{2\sqrt{c + 6}} $$
    $$\lim_{n \to \infty}\abs{\frac{{x_{n + 1} - c}}{x_{n} - c}} =  \frac{1}{2c}, c = 3 $$
    $$\lim_{n \to \infty}\frac{|{x_{n + 1} - c}|}{|x_{n} - c|^{\alpha}} =  \frac{1}{2c} = \frac{1}{6} = \lambda $$
    This shows that the rate of convergence $\alpha$ is $\alpha = 1$
\end{enumerate}

% PROBLEM 3
    \item 
    \begin{enumerate}
        \item Give Taylor's theorem for a function f at a point $x_0$.\\
        The following is necessary in order to apply Taylor's theorem.\\
        \textit{$f \in C^k[a,b], f^{k +1}$ exist on given interval and $x_0 \in [a,b]$ there exists a number $\xi(x)$ between $x_0$ and $x$.}\\
        $$f(x) = \sum_{k = 0}^{\infty} \frac{f^{(k)}(x_0)(x - x_0)^k}{k!} + \frac{f^{(k+1)}(\xi(x))}{(k+1)!}(x - x_0)^{k+1}$$
        \item Using Newton's method, what is the rate of convergence for the sequence error?\\
        \\
        The rate of convergence for Newton's method is Quadratic.\\=
        %
        \item Give the largest interval possible approximation of a number p up to $10^{-5}$ relative error.\\
        We know that relative error $\frac{|p-p*|}{p}$
        $$\frac{|p-p*|}{p} < 10^{-5}$$
        $$|p - p*| < p\times10^{-5}$$
        $$-p\times10^{-5} < p - p* < p\times10^{-5}$$
        $$-p\times10^{-5} + p < p* < p\times10^{-5} + p$$
        The largest interval $p*$ must lie on is $ (p(-10^{-5} + 1),p(10^{-5} + 1) )$
        \item Suppose $p*$ approximates the number p. What are the different types of error that you can use to quantify the accuracy of the approximation.\\
        \begin{enumerate}
            \item Relative error:  $\abs{\frac{p-p*}{p}}$
            \item Absolute error:  $\abs{p-p*}$
            \item Actual error:    $p-p*$
        \end{enumerate}
    \end{enumerate}
    
% PROBLEM 4
    \item Below is a MATLAB implementation of the bisection method with some code missing.
    \begin{enumerate}
    \item Fill in the blanks to correctly execute the bisection method.
    \begin{verbatim}
    function [c,err,n] = bisection(f,a,b,tol,N)
    %function to solve f(x) = 0 using the bisection method over [a,b]
    %ASSUMPTIONS: we assume f(a)*f(b) < 0
    %INPUTS: 
    %f is a function at hand
    % a is the lower bound of the tested interval
    % b is the upper bound of the tested interval
    % tol is the error tolerance
    % N is the maximum number of iterations
    %OUTPUTS:
    % c is the computed root
    % err is the error bound at the end
    % n is the last ieration before breaking
        n = 0;
        a_n = a;
        b_n = b;
        err = b_n - a_n;
        while err > tol && n < N
            x_n = %(a_n + b_n)/2;
            if f(x_n)*f(a_n) > 0
                %a_n = x_n;
                %c = a_n;  <--Consider revising to [b_n = b_n] based on previous semester.
            elseif f(x_n)*f(b_n) >= 0
                %b_n = x_n;
                %c = b_n;  <--Consider revising to [a_n = a_n] based on previous semester.
            end
        err = (b_n - a_n)/2;
        n = n + 1;
    end
    c = (b_n + a_n)/2.0;
end
    \end{verbatim}
    
    \item Is the code working if you consider $f(x) = x^4, a = -1, b = 2$? Explain why.\\\\
    No. The bisection method has been established by the intermediate value theorem which states if the a function $f(x)$ is continuous in a given interval ([a,b]) and if evaluated endpoints $f(a)$ and $f(b)$ have opposite signs that is $f(a)\times f(b) < 0$ there there exists a root inside the interval. $$f(-1) = 1, f(2) = 2^4, f(-1)\times f(2) \not < 0$$ 
    \end{enumerate}
\end{enumerate}
\end{document}