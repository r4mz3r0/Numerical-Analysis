\documentclass{article}
\usepackage[utf8]{inputenc}
\usepackage{amsmath}
\usepackage{booktabs}
\usepackage{color}
\usepackage{lipsum}
\usepackage{amsmath}
\usepackage{amssymb}
\usepackage{fancyhdr}
\usepackage[T1]{fontenc}
\usepackage{fourier}
\usepackage{listings}
% document size parameters

\setlength{\oddsidemargin}{0.0in}
\setlength{\evensidemargin}{0.0in}
\setlength{\textheight}{8.4in}
\setlength{\textwidth}{6.5in}
\setlength{\voffset}{-0.40in}
\setlength{\headsep}{26pt}
\setlength{\parindent}{0pt}
\setlength{\parskip}{6pt}

% header information
\pagestyle{fancyplain}
\lhead{\large{{\bf  Numerical Analysis} }}
\rhead{\large{{\bf }}}
\title{Chapter 1: Mathematical Preliminaries Exercises}
\author{Ramiro Gonzalez }
\date{}
\begin{document}
\maketitle
\section{Exercise Set 1.1}
\begin{enumerate}
    \item Show that the following equations have at least one solution in the given intervals.
    \begin{enumerate}
        \item $xcos(x)-2x^2+3x-1 = 0, [0.2,0.3]$ and $[1.2,1.3]$\\
        \begin{enumerate}
            \item $[0.2,0.3]$ Let $f(x) = xcos(x) - 2x^2 + 3x - 1$ we know that it is continuous, for the interval $[0.2,0.3]$ \\
                $f(0.2) = .2cos(.2)-2(.2)^2+3.2-1 = -0.2840 < 0 $ and $f(.3) = .3cos(.3)-2(.3)^2+3.3-1 = 0.0066 > 0  $\\
                By the intermediate value theorem we know that a number c such that $.2 < c < .3$ exists. There is at least one solution c such $(c)cos(c)-2c^2 + 3c - 1 = 0$. 
            \item  f(x) is continuous on the interval $[1.2,1.3]$, we find $f(1.2) = 0.1548 > 0 $ and $f(1.3) = -0.1323 < 0 $ this means $f(1.2) > f(1.3)$. From the intermediate value theorem we know that there exist an c such that $1.2 < c < 1.3$ for $f(c) = 0$, so there is at least one solution in the interval $[1.2,1.3]$
        \end{enumerate}
        \item $(x-2)^2-ln(x)=0, [1,2]$ and $[e,4]$
        \begin{enumerate}
            \item Let $f(x) = (x-2)^2-ln(x) = 0$ we want to show that there exists at least one solution in the interval $[1,2]$. $f(1) = 1 > 0, f(2) = -ln(2) <0$ that is $f(1) > f(2)$. F(x) is continuous, By the intermediate value theorem there exist a c such that $1 < c < 2$ for $f(c) = 0$.
            \item $f(x)$ is continuous in the interval $[e,4]$ and $f(e) = (e-2)^2-ln(e) = -0.4841 < 0, f(4) = (4-2)^2-ln(4) = 2.6137 > 0, f(e) < f(4)$ By the intermediate value theorem there exist c such that $e < c < 4$.
        \end{enumerate}
        \item $2xcos(2x)-(x-2)^2=0, [2,3]$ and $[3,4]$
        \begin{enumerate}
            \item Let $f(x) = 2xcos(2x)-(x-2)^2=0$, $f(x)$ is continuous, $f(2)= 4cos(4) = -2.6146 < 0, f(3) = 6cos(6)-(1) = 4.7610 > 0 $ If $f\in C[a,b]$ and K is any number between $f(a)$ and $f(b)$ then there exist a number c in $(a,b)$ for which $f(c) = K$. 
            Since $f(x)$ is continuous and $f(2) < f(3)$, where $f(x) = 0$ then there exist at least one solution c such that $f(c) = 0$
            \item We know that $f(x)$ is continuous, $f(3) =4.7610 > 0, f(4) = -5.1640 < 0 $ therefore $f(3) > f(4)$ where $f(c) = K = 0$ so there exist (by the intermediate value theorem) a number c $3 < c < 4$ for $f(c) = 0$.
        \end{enumerate}
        \item $x-ln(x)^x = 0, [4,5]$ 
        \begin{enumerate}
            \item Let $f(x) = x - ln(x)^x $, $f(x)$ is continuous,$f(4) = 4 - ln(4)^4 = 0.3066 > 0, f(5) = -5.7987 < 0 $, $f(4) > f(5)$ it is decreasing. By the intermediate value theorem there exist a number c such that $4 < c < 5$ for $f(c) = 0$.
        \end{enumerate}
    \end{enumerate}
    \item Show that the following equations have at least one solution in the given interval.
    \begin{enumerate}
        \item $f(x) = \sqrt{x}-cos(x) = 0, [0,1]$\\
        $f(x)$ is continuous on the interval $[0,1]$, $f(0)=-1 < 0, f(1) = 1 - cos(1) = 0.4597 > 0$, therefore $f(0) < f(1) $by the intermediate theorem we know there exist a number c such that $0 < c < 1$ for $f(c) = 0$.
        \item $f(x) e^{x} - x^2 + 3x - 2 = 0, [0,1]$ we know that $f(x) \in C[0,1]$  and $f(0) = -1 < 0, f(1) = e > 0, f(0) < f(1) $ By the intermediate value theorem there exists a number c such that $0 < c < 1$ such that $f(c) = 0$
        \item $f(x) = -3tan(2x)+x=0, [0,1]$, $f(x)$ is continuous, $f(0) = 0, f(1) = -3.6722, f(1) < f(0)$ since the inteval $[0,1]$ is closed, there exist a number c such that $0 < c < 1$ where that c = 0, for $f(c) = 0$
        \item $f(x) = ln(x) - x^2 + \frac{5}{2}x - 1 = 0, [\frac{1}{2},1]$ f(x) is continuous in the given interval $f(1/2) = ln(\frac{1}{2}) - (\frac{1}{2})^2 = -0.6931 < 0, f(1) =  0.5000 > 0, f(1/2) < f(1)$ By the intermediate value theorem we know that there exist a numbe c such that $\frac{1}{2} < c < 1$ exist for $f(c) = 0$
    \end{enumerate}
    \item Find intervals containing solutions to the following equations. 
    \begin{enumerate}
        \item $f(x) = x - 2^{-x} = 0$ f(x) is continuous. From the intermediate value theorem we know $f(0) = -1 < 0, f(1) = \frac{1}{2}$ so the value c exists between $[0,1]$.
        \item $2xcos(2x) - (x+1)^2 = 0$, $f(-1) = .83$, and $f(0) = -1$, f(x) is continuous, and a number c exist in the interval $[-1,0]$
        \item $f(x) = 3x-e^{x} = 0$ we know that f(x) is continuous everywhere $f(0) = -1 < 0, f(1) = 0.2817 > 0$ in interval $[0,1]$ there exist a number c such that $f(c) = 0$.
        \item $f(x) = x + 1 - 2sin(\pi x) = 0$ $f(\frac{-3}{2}) = \frac{-5}{2}, f(\frac{-1}{2} = \frac{5}{2}$ therefore there exists (by the intermediate value theorem) a number c where $\frac{-3}{2} < c < \frac{-1}{2}$ for which $f(c) = 0$
    \end{enumerate}
    \item Find intervals containing solutions to the following equations.
    \begin{enumerate}
        \item $f(x) = x-3^{-x} = 0$, $f(0) = -3 < 0, f(1) = 1 - 3^{-1} = 0.6667 > 0, f(0) < f(1)$, By the intermediate value theorem there exists a number c such $0 < c < 1$, for $f(c) = 0$.
        \item $f(x) = 4x^2-e^{x} = 0$, $f(0) = -1 < 0, f(1) = 4-e > 0$ therefore $f(0) < f(1)$ BY the intermediate value theorem there exist a number c such that $0 < c < 1$ for $f(c) = 0$.
        \item $f(x) = x^3-2x^2-4x+2 = 0$, $f(0) = 2 > 0, f(1) = -3 < 0$ therefor $f(0) < f(1)$ By the intermediate value theorem there exist a number c such that $0 < c < 1$ for $f(c) = 0$.
        \item $f(x) = x^3+4.001x^2 + 4.002x + 1.101 = 0$, $f(-3) = -1.8960 < 0, f(-2) = 1.1010, f(-3) < f(-2) $ By the intermediate value theorem we know that $f \in C[-3,-2]$ and there exist a number c such that $-3 < c < -2$ for $f(c) = 0$
    \end{enumerate}
    \item Find $max_{a \leq x \leq b}|f(x)|$ for the following functions and intervals.
    \begin{enumerate}
        \item $f(x) = \frac{(2-e^{x}+2x)}{3}, [0,1]$, $f'(x) = \frac{-e^{x} + 2}{3}, f'(x) = 0, x = ln(2) $ $f(x)$ us continuous everywhere, critical values, $\{0,ln(2),1\}$
        $$f(0) = \frac{1}{3} = 0.3333, f(ln(2)) = \frac{(2)(ln(2))}{3} = 0.4621, f(1) = \frac{4-e}{3} = 0.4272$$
        $max_{0 \leq x \leq 1}|\frac{(2-e^{x}+2x)}{3}| \approx .4621,$ where x = ln(2)
        \item $f(x) = \frac{4x-3}{x^2-2x}, [0.5,1]$ First we must determine if $f(x)$ is continuous on interval given. $4x-3$ is continuous everywhere, $x^2-2x = 0, x = 0, x = 2$ outside of the interval given. $f'(x) = \frac{(x^2-2x)(4)-(4x-3)(2x-2)}{(x^2-2x)^2} = 0,  (-2x^2+3x-3) = 0 $, no real solutions, thus we are left with the interval endpoints. 
        $f(.5) = 1.3333, f(1) = -1$ therefore $max_{.5 \leq x \leq 1}|\frac{4x-3}{x^2-2x}| = 1.3333$ where x = .5.
       \item $f(x) = 2xcos(2x) - (x-2)^2, [2,4]$, $f(x)$ is continuous everywhere, next we differentiate, $f'(x) = cos(2x) - 4xsin(x)cos(x) - (x-2) =0$, $f'(x) = 0,f(2) = -2.6146, f(4) = -5.1640$ it follows $max_{2 \leq x \leq 4} |f(x)| = 5.1640$ where x = 4
        \item $f(x) = 1 + e^{-cos(x-1)}, [1,2]$ f(x) is continuous everywhere, $f'(x) = sin(x-1)e^{-cos(x-1)} = 0, x =1, f(1) = 1.3679, f(2) = 1.5826 $ therefore $max_{1 \leq x \leq 2}|1 + exp(-cos(x-1)) \approx 1.5826$
    \end{enumerate}
    \item Find $max_{a \leq x \leq b}|f(x)|$ for the following functions and intervals. 
    \begin{enumerate}
        \item $f(x) = \frac{2x}{x^2+1}, [0,2]$, we can see that $2x$ is continuous, while $x^2+1$ is not continuous at $x = \pm i$ where $\pm i \not \in [0,2]$. $f'(x) = \frac{(2x^2+2) - (4x^2)}{(x^2+1)^2}, f'(x) = 0,2 - 2x^2 = 0, x = \pm 1, -1 \not \in [0,2]$, $f(0) = 0, f(2) = \frac{4}{5}$ therefore $max_{0 \leq x \leq 2}|f(x = 1)| = 1$
        \item $f(x) = x^2\sqrt{4 - x},[0,4]$ while $4-x > 0$ $f(x)$ is continuous, $f'(x) = 2x\sqrt{4-x} + x^2(-\frac{1}{2\sqrt{4-x}}) = 0, x = 0,x = \frac{16}{5} = 3.2$, critical values $0,16/5,4$, $f(0) = 0, f(16/5) = 9.16, f(4) = 0$ therefore $max_{0 \leq x \leq 4 }|f(x = 16/5)| = 9.16$.
        \item $f(x) = x^3 - 4x + 2, [1,2]$, f(x) is continuous everywhere, $f'(x) = 3x^2-4, x = \sqrt{4/3}$, . Therefore $|f(1)| = 1, f(\sqrt{4/3}) = 1.08,f(2) = 2$ this shows $max_{1 \leq x \leq 2} |f(x = 2)| = 2$
        \item $f(x) = x\sqrt{3-x^2}, [0,1]$, $f(x)$ is continuous on the interval. $f'(x) = \sqrt{3-x^2} + x\frac{-2x}{2\sqrt{3-x^2}} = \frac{3-2x^2}{\sqrt{3-x^2}}, f'(x) = 0$, $x = \pm \sqrt{3} \not \in [0,1]$, $f(0) = 0, f(1) = \sqrt{2}, max_{0 \leq x \leq 1}|f(x = 1)| = \sqrt{2}$
    \end{enumerate}
    \item Show that $f'(x)$ is 0 at least once in the given intervals. 
    \begin{enumerate}
        \item $f(x) = 1 - e^x + (e-1)sin((\frac{\pi}{2})x), [0,1]$ Taking into consideration rolle's theorem, we must show that $f\in [0,1]$, $f$ i differentiable in $[0,1]$, and if $f(0) = f(1)$, then there exist a number c such that $c \in [0,1]$ for $f'(c) = 0$. $f'(x) = -e^x + (\frac{\pi}{2})(e-1)cos(\frac{\pi}{2}x)$ there are no singularities in $[0,1]$ therefore $f$ is differentiable in that interval, $f(0) = 0, f(1) = 0, f(0) = f(0)$, by rolle's theorem there exist a number c $\in [0,1]$ such that $f'(c) = 0$
        \item $f(x) = (x-1)tan(x) + xsin(\pi x), [0,1]$, $f'(x) = (x-1)sec^2(x) + tan(x) + x\pi cos(\pi x) + sin(\pi x)$, $f'(x)$ is continuous, no singularities therefore $f(x)$ is differentiable and thus continuous on the interval. $f(0) = 0, f(1) = 0, f(0) = f(1)$ therefore f(x) satisfies Rolle's theorem which means "there exists at least one c in (0,1) such that $f'(c) = 0$.
        \item $f(x) = (x)sin(\pi x) - (x-2)ln(x), [1,2]$, $f'(x) = sin(\pi x) + x\pi cos(x) -\frac{(x-2) + xln(x)}{x}$, $f'(x)$ has a singularity at x = 0, where $0 \not \in [1,2]$ therefore $f(x)$ is differentiable and continuous on $[1,2]$, $f(1) = 0 , f(2) = 0, f(1) = f(2)$ by rolle's theorem we can conclude there exist a number c such that $c \in (1,2)$ where $f'(c) = 0$.
        \item $f(x) = (x-2)sin(x)ln(x+2), [-1,3]$, $f'(x) =sin(x)ln(x+2) + (x-2)cos(x)ln(x+2) + (x -2)sinx(1/(x+2))$, singularity at x = -2, no in the interval, f(x) is differentiable and continuous, $f(-1) = 0, f(3) = 1, f(-1) \neq f(3)$ therefore f(x) fails to satisfy the Rolle's Theorem. We can not conclude there is a number $c \in [-1,3], f'(c) = 0$.
    \end{enumerate}
    \item Suppose $f\in C[a,b]$ and $f'(x)$ exists on (a,b). Show that if $f'(x) \neq 0$ for all x in (a,b), there there can exist at most one
    number p in [a,b] with f(p) = 0. .
    \item Let $f(x) = x^3$
    \begin{enumerate}
        \item Find the second Taylor polynomial $P_2(x)$ about $x_0 = 0$.
        $$P_n(x) = \sum_{k=0}^{n} \frac{f^{k}(x_0)}{k!}(x-x_0)^{k}$$
        $$f(0) = 0, f^{1} = 3x^2 = 0, f^{2} = 6x = 0$$
        $$P_2(0) = \frac{f(0)(x-0)^{0}}{0!} + \frac{f^{1}(0)(x-0)^{1}}{1!} + \frac{f^{2}(0)(x-0)^{2}}{2!} = 0$$
        \item Find $R_2(.5)$ and the actual error in using $P_2(.5)$ to approximate $f(.5)$.\\Recall $x_0 = 0$
        $$R_n(x) = \frac{f^{n+1}(\rho (x))}{(n+1)!}(x-x_0)^{n+1}$$
        $$R_2(.5) = \frac{f^{3}(\rho(x))}{(3)!}(x-0)^{3}, f^{3} = 6$$
        $$R_n(x) = \frac{f^{3}(\rho(x))}{3!}(x-0)^3$$
        $$R_n(x) = \frac{6(x^3)}{6}$$
        $$R_2(0.5) = .125$$
        \item Repeat part (a) using $x_0 = 1$.
        $$P_n(x) = \sum_{k=0}^{n} \frac{f^{k}(x_0)}{k!}(x-x_0)^{k}$$
        $$f(1) = 1 , f^{1} = 3, f^{2} = 6$$
        $$P_2(1) = \frac{f(1)(x-1)^{0}}{0!} + \frac{f^{1}(1)(x-1)^{1}}{1!} + \frac{f^{2}(1)(x-1)^{2}}{2!} = 1 + 3(x-1) + 3(x-1)^2$$
        \item Repeat part (b) using the polynomial form part (c).
        $$R_n(x) = \frac{f^{3}(\rho(x))}{3!}(x-1)^3, x_0 = 1$$
        $$R_n(x) = (x-1)^3$$
        $$R_2(.5) = (.5-1)^3 = -0.125$$
        $$f(x) = x^3$$
        $$f(.5) = .125, f(0.5) - P_2(.5) = -.125$$
    \end{enumerate}
    \item Find the third Taylor polynomial $P_3(x)$ for the function $f(x) = \sqrt{x+1}$ about $x_0 = 0$. Approximate $\sqrt{.5}. \sqrt{.75},\sqrt{1.25},\sqrt{1.5}$ using $p_3(x)$ and find the actual errors.
    $$f(x) = \sqrt{x+1},f'(x) = \frac{1}{2\sqrt{x+1}}, f^{2} =\frac{-1}{4(x+1)^{-3/2}} , f^{3} = \frac{3}{8(x+1)^{-5/2}}$$
    $$f(0) = 1,f'(0) = \frac{1}{2}, f^{2}(0) =\frac{-1}{4} , f^{3}(0) = \frac{3}{8}$$
    $$P_3(x) = \frac{(1)(x-0)^{0}}{0!} + \frac{(1/2)(x-0)^1}{1!} + \frac{(-1/4)(x-0)^2}{2!} + \frac{(3/8)(x-0)^3}{3!}$$
    $$P_3(x) = 1 + \frac{x}{2} - \frac{x^2}{8} + \frac{x^3}{16}$$
    $$P_3(.5) = 1.2265, P_3(.75) = 1.331, P_3(1.25) = 1.5518 , P_3(1.5) = 1.679$$
    For actual error, $|\sqrt{.5 + 1} - P_3(.5)| = .00018$, 
    $|\sqrt{.75 + 1} - P_3(.75)| = .0082$, 
    $|\sqrt{1.25 + 1} - P_3(1.25)| = .052$,
    $|\sqrt{1.5 + 1} - P_3(1.5)| = .099$
    \item Find the second Taylor polynomial $P_2(x)$ for the function $f(x) = e^{x}cos(x)$ about $x_0 = 0$.
    \begin{enumerate}
        \item Use $P_2(.5)$ to approximate $f(.5)$. Find an upper bound for error $|f(.5) - P_2(.5)|$ using the error formula and compare it to the actual error.  $$P_n(x) = \sum_{k=0}^n\frac{f^{k}(x_0)}{k!}(x-x_0)^k$$
        $$f(x) = e^{x}cos(x), f^{1}(x) = -e^{x}sin(x) + e^{x}cos(x),f^{2}(x) = -2e^{x}sin(x) $$
        $$P_2(x) =\frac{f(0)}{0!}(x-0)^0 +  \frac{f^{1}(0)}{1!}(x-0)^1 + \frac{f^{2}(0)}{2!}(x-0)^2$$
        $$f(0) = 1, f'(0) = 1, f''(0) = 0, P_2(x) = 1 + x$$
        $$P_2(.5) = 1.5, f(.5) = e^{.5}cos(.5) =1.4469 $$
        $$|1.4469 - 1.5| = 0.0531$$
        Next we find $R_2(x = .5)$,$R_n(x) = \frac{f^{n+1}(\xi(x))}{(n+1)!}(x-x_0)^{n+1}$
        $$R_2(x) = \frac{f^3(\xi(x))}{3!}(x-0)^3$$
        $$R_2(x) = \frac{-e^{\xi(x)}[cos(\xi(x)) - sin(\xi(x))]}{3}(x)^3$$
        $$R_2(.5) = \frac{-e^{\xi(.5)}[cos(\xi(.5)) - sin(\xi(.5))]}{3}(.5)^3$$
        $$|R_2(.5)| \leq max_{\xi(x)\in[0,0.5]}|\frac{-e^{\xi(.5)}[cos(\xi(.5)) - sin(\xi(.5))]}{3}(.5)^3|$$
        $$|R_2(.5)| \leq \frac{(.5)^3}{3} max_{\xi(x)\in[0,0.5]}|e^{\xi(.5)}[cos(\xi(.5)) - sin(\xi(.5))]|$$
        $$|f(0.5) - P_2(.5)| \leq ((.5)^3/3)(2.24) \leq .0933333$$
        The actual error is $= .0531$
        \item Find a bound for the error $|f(x) - P_2(x)|$ in using $P_2(x)$ to approximate $f(x)$ on the interval $[0,1]$
        $$R_2(x) = \frac{-e^{\xi(x)}[cos(\xi(x)) - sin(\xi(x))]}{3}(x)^3$$
        \item Approximate $\int_0^1f(x)dx$ using $\int_0^1 P_2(x)dx$.
        $$\int_0^1f(x)dx, = \frac{2x + x^2}{2}|_0^1 \approx 1.5$$
        \item Find an upper bound for the error in (c) using $\int_0^1|R_2(x)dx|$ and compare the bound to the actual error. $.121975486$
    \end{enumerate}
    \item Repeat Exercise 11 using $x_0 = \frac{\pi}{6}$
    \item Find the third Taylor polynomial $P_3(x)$ for the function $f(x) = (x-1)ln(x)$ about $x_0=1$.
    \begin{enumerate}
        \item Use $P_3(.5)$ to approximate $f(.5)$. Find an upper bound for error $|f(.5) - P_3(.5)|$ using error formula and compare it to the actual error.$$P_3(x) = (x-1)^-\frac{1}{2}(x-1)^3, P_3(.5) = .3125, f(.5) = .346573,|f(.5)-P_3(.5)|=|R_3(.5)||$$
    \end{enumerate}
    \item Let $f(x) = 2xcos(2x) - (x-2)^2, x_0 = 0$
    \begin{enumerate}
        \item 
    \end{enumerate}
    \item Find the fourth Taylor polynomial $P_4(x)$  for the function $f(x) = xe^{x^2}$ about $x_0 = 0$
    \begin{enumerate}
        \item 
    \end{enumerate}
    Use the error term of a Taylor polynomial to estimate the error involved in using $sin(x) \approx x$ to approximate $sin(1^o)$
    \item Use a Taylor polynomial about $\pi/4$ to approximate $cos(42^o)$ to an accuracy of $10^{-6}$.
    \item Let $f(x) = (1-x)^{-1}$ and $x_0 = 0$. Find the nth Taylor polynomial $P_n(x)$ for $f(x)$ about $x_0$. Find a value of n necessary for $P_n(x)$ to approximate f(x) to within $10^{-6}$ on [0,.5].
    \item Let $f(x) = e^x$ and $x_0 = 0$. Find the nth Taylor polynomial $P_n(x)$ for $f(x)$ about $x_0$. Find a value of n necessary for $P_n(x)$ to approximate f(x) to within $10^{-6}$ on [0,.5].
    \item Find the nth Maclaurin polynomial $P_n(x)$ for $f(x) = arctan(x)$.
    \item The polynomial $P_2(x) = 1 - \frac{1x^2}{2}$ is to be used to approximate $f(x) = cos(x)$ in $[\frac{-1}{2},\frac{1}{2}]$. Find a bound for the maximum error. 
    \item Use the Intermediate Value Theorem 1.11 and Rolle's Theorem 1.7 to show that the graph of $f(x) = x^3 + 2x + k$ crosses the x-axis exactly once, regardless of the value of the constant k.
    \item A Maclaurin polynomial for $e^x$ is used to give the approximation 2.5 to e. The error bound in this approximation is established to be $E=\frac{1}{6}$. Find a bound for the error in E.
    \item The error function defined by
        $$erf(x) = \frac{2}{\sqrt{\pi}}\int_{k=0}^{x}e^{-t^2}dt$$
    gives the probabilty that any one of a series of trials will ie within x units of the mean, assuming that the trials have a normal distribution mean 0 and standard deviation $\frac{\sqrt{2}}{2}$. This integral cannot be evaluated in terms of elementary functions, so an approximating technique must be used. 
    \begin{enumerate}
        \item Integrate the Maclaurin series for $e^{-x^2}$ to show that
        $$erf(x) = \frac{2}{\sqrt{\pi}}\sum_{k=0}^{\infty}\frac{2^{k}x^{2k+1}}{(2k+1)k!}$$
        \item The error function can also be expressed in the form
        $$erf(x) = \frac{2}{\sqrt{\pi}}\sum_{k=0}^{\infty}\frac{2^{k}x^{2k+1}}{1\cdot3\cdot5\cdot\cdot(2k+1)}$$
        Verify that the two series agree for $k = 1,2,3, 4$. [hint: Use the Maclaurin series for $e^{-x^2}$].
        \item Use the series in part (a) to approximate $erf(1)$ to within $10^{-7}$.
        \item Use the same number of terms as in part (c) to approximate erf(1) with the series in part (b).
        \item Explain why the difficulties occur using the series in part(b) to approximate erf(x).
    \end{enumerate}
\end{enumerate}
\section{Exercise Set 1.2}
\begin{enumerate}
    \item Compute the absolute error and relative error in approximations of p by p*.
    \begin{enumerate}
        \item $p = \pi, p* = 22/7$
        Actual Error $p = p*$, absolute error $|p - p*|$, and relative error is $\frac{|p-p*|}{|p|}$,
        provided that $p\neq 0$. Absolute Error: $|\pi - 22/7| = 0.0013 $
        Relative Error: $\frac{|\pi - 22/7|}{|\pi|} = .00040250$
        \item $p = \pi, p* = 3.1416$
        Absolute: $|\pi - 3.1416| =.0000073464$\\
        Relative: $\frac{|\pi - 3.1416|}{|p|} =\frac{.0000073464}{\pi} = 2.3384e-06$
        \item $p = e, p* = 2.718$\\
        Absolute: $|e - 2.718| =0.4236 $\\
        Relative: $\frac{|e-2.718|}{e} =1.0368e-04 $\\
        \item $p = \sqrt{2}, p*=1.414$\\
        Absolute: $|\sqrt{2}-1.414| =  2.1356e-04 $\\
        Relative: $\frac{|\sqrt{2}-1.414|}{\sqrt{2}} =1.0368e-04$
    \end{enumerate}
    \item Compute the absolute error and relative error in approximations of p by p*.
    \begin{enumerate}
        \item Absolute: $|e^{10} - 22000| = 26.4658$\\
        Relative: $\frac{|e^{10} - 22000|}{e^{10}} = 0.0012$
        \item Absolute: $|10^{\pi} - 1400| = 14.5443 $\\
        Relative: $|10^{\pi} - 1400|/|10^{\pi}| = 0.0105$
        \item $p = 40320, p* = 39900$\\
        Absolute: $|40320-39900| = 420$\\
        Relative: $|40320-39900|/|40320| = 0.0104$
        \item $p = 9!, p* = \sqrt{18\pi}(9/e)^{9}$\\
        Absolute: $|362880-\sqrt{18\pi}(9/e)^{9}| =3.3431e+03 $\\
         Relative: $|362880-\sqrt{18\pi}(9/e)^{9}|/|362880| = 0.0092$
    \end{enumerate}
    \item Suppose p* must approximate p with relative error at most $10^{-3}$. Find the largest interval in which p* must lie for each value of p. 
    \begin{enumerate}
        \item 150
        $$\frac{|150 - p*|}{|150|} = 10^{-3}$$
        $$|150 - p*| = 10^{-3}(150)$$
         $$-150(10^(-3)) < (150 - p*) < 10^{-3}(150)$$
         $$150(10^(-3)) + 150 < p* < -10^{-3}(150) + 150$$
         $$150((10^(-3)) + 1)< p* < 150(-10^{-3} + 1)$$
        \item 900
        $$\frac{|900 - p*|}{|900|} = 10^{-3}$$
        $$-900(10^(-3)) < (900 - p*) < 10^{-3}(900)$$
        $$900(10^(-3)) + 900 < p* < -10^{-3}(900) + 900$$
        \item 1500
        $$\frac{|1500 - p*|}{|1500|} = 10^{-3}$$
        $$-1500(10^(-3)) < (1500 - p*) < 10^{-3}(1500)$$
        $$1500(10^(-3)) + 1500 < p* < -10^{-3}(1500) + 1500$$
        \item 90
        $$\frac{|90 - p*|}{|90|} = 10^{-3}$$
        $$-90(10^(-3)) < (90 - p*) < 10^{-3}(90)$$
        $$90(10^(-3)) + 90 < p* < -10^{-3}(90) + 90$$
    \end{enumerate}
    \item Suppose p* must approximate p with relative error at most $10^{-3}$. Find the largest interval in which p* must lie for each value of p.
\end{enumerate}
\section{1.3}
\section{1.4}



\end{document}
