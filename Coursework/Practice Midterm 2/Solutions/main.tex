\documentclass{article}
\usepackage[utf8]{inputenc}
\usepackage{amsmath}
\usepackage{booktabs}
\usepackage{color}
\usepackage{lipsum}
\usepackage{amsmath}
\usepackage{amssymb}
\usepackage{fancyhdr}
\usepackage[T1]{fontenc}
\usepackage{fourier}
\usepackage{listings}
\usepackage{graphicx}
\usepackage{physics}
\usepackage{listings}

% document size parameters

\setlength{\oddsidemargin}{0.0in}
\setlength{\evensidemargin}{0.0in}
\setlength{\textheight}{8.4in}
\setlength{\textwidth}{6.5in}
\setlength{\voffset}{-0.40in}
\setlength{\headsep}{26pt}
\setlength{\parindent}{0pt}
\setlength{\parskip}{6pt}

% header information
\pagestyle{fancyplain}
\lhead{\large{{\bf  Numerical Analysis} }}
\rhead{\large{{\bf Solutions}}}
\title{Practice Midterm II}
\author{Ramiro Gonzalez }
\date{}
\begin{document}
\maketitle
\textbf{\Large{Formulas:}}
\color{blue}
\begin{enumerate}
    \item[F1]Composite Trapezoidal
    $$h = \frac{b - a}{n}$$
    $$\frac{h}{2}\Big[f(a) + 2\cdot \sum_{j = 1}^{n-1}f(x_j) + f(b)\Big] - \frac{b-a}{12}h^2f^{(2)}(\xi)$$
    \item[F2] Composite Simpsons 
    $$S = \frac{h}{3}\Big[f(a) + 2\cdot \sum_{j = 1}^{\frac{n}{2} - 1} f(x_{2j}) + 4\codt \sum_{j = 1}^{\frac{n}{2}}f(X_{2j - 1}) + f(b)\Big] - \frac{b - a}{180}h^4f^{(4)}(\xi)$$
    \item[F2] Midpoint Formula
    $$h = \frac{b - a}{n + 2}$$
    $$2\cdot h\Big[2\sum_{j = 0}^{n/2}f(x_{2j})\Big] - \frac{b - a}{6}h^2f^{(2)}(\xi)$$
\end{enumerate}
\color{black}
\begin{enumerate}
    \item (20 pts, 10 each)
    \begin{enumerate}
%Problem 1
        \item Given $x_0 = 2, x_1 = 3, f(x_0) = 2, f(x_1) = 3,$ construct the linear Lagrange Polynomial interpolant of f that passes through the points $((x_0,f(x_0),(x_1,f(x_1))$. To get full credit, you must give the Lagrange polynomials you need.\\
        \\
        \color{red}
        We must find the interpolation polynomial of degree 1, given that it is linear and we are given 2 nodal points. To find degree $n = 1$, we need $n + 1$ points. 
        \begin{enumerate}
            \item Recall that the Lagrange interpolating polynomial is as follows. 
            $$P(x) = \sum_{k = 0}^{n} f(x_k)L_{n,k}(x)$$
            \item We know that $y_{0} = f(x_0) = 2, y_{1} = f(x_1) = 3$
            \item  find $L_{n,k}$
            $$L_{1,0} = \frac{x - x_{1}}{x_{0} - x_{1}}, L_{1,1} = \frac{x - x_0}{x_1 - x_0}$$
            $$L_{1,0} = \frac{x - 3}{2 - 3} = -(x - 3) , L_{1,1} = \frac{x - 2}{3 - 2} = (x - 2)$$
            \item $P_{1}(x) = y_{0}L_{1,0} + y_{1}L_{1,1}$
            $$P_1(x) = 2\cdot(-(x - 3)) + 3\cdot(x - 2) = x $$
             \end{enumerate}
            We only need one Lagrange polynomial, we are only given two nodal points, making it linear Lagrange interpolation
            \color{black}
            \item Use Newton's divided difference formula to construct an interpolating polynomial of degree two of f for the following data: $(x_{0},f(x_{0}) = (-1,1), (x_1, f(x_1)) = (0,.5), (x_2,f(x_2)) = (1,3)$.\\
            \\
            \color{red}
            We want to find the interpolating polynomial of degree 2. 
            \begin{enumerate}
                \item Consider the following form of Newton's Divided Difference:
                $$f[x_{i},x_{i + 1}] = \frac{f[x_{i + 1}] - f[x_{i}]}{x_{i + 1} - x_{i}}$$
                $$P_{n}(x) = f[x_0] + \sum_{k = 1}^{n} f[x_0,x_1,\dots x_k](x-x_0)\dots(x - x_{k -1})$$
                Let n = 2, we want to find the following. 
                $$P_{2}(x) = f[x_0] + f[x_0,x_1](x-x_0) +  f[x_0,x_1,x_2](x-x_0)(x - x_{1}) $$
                \item Find $f[x_0,x_1]$
                $$f[x_0,x_1] = \frac{f[x_1] - f[x_0]}{x_1 - x_0}$$
                \item Find $f[x_0,x_1,x_2]$\\
                Recall that $f[x_0,x_1]$
                $$f[x_0,x_1] = \frac{f[x_1] - f[x_0]}{x_1 - x_0}$$
                Recall that $f[x_1,x_2]$
                $$f[x_1,x_2] = \frac{f[x_{2}] - f[x_{1}]}{x_{2} - x_{1}}$$
                Therefore 
                $$f[x_0,x_1,x_2] = \frac{f[x_1,x_2] - f[x_0,x_1]}{x_2 - x_0}$$
                
                \item Consider three nodal points, that is $x_0, x_1, x_2$, if n = 2, then we need $n + 1$ points. 
                $$x_0 = -1, x_1 = 0, x_2  = 1$$
                \item We define values of the nodal points at $f$. It is already given
                $$f[x_0] = 1, f[x_1] = .5, f[x_2] = 3$$
                \item Using the form of Newton's divided difference. Substituting previous findings. 
                $$f[x_0,x_1] = \frac{.5 - 1}{0 - (-1)} = -.5$$
                $$f[x_1,x_2] = \frac{3 - .5}{1 - 0} = 2.5$$
                $$f[x_0,x_1,x_2] = \frac{2.5 - .5}{1 - (-1)} = 1$$
                \item Finding $P_2$. Substituting the values we found. 
                $$P_{2}(x) = f[x_0] + f[x_0,x_1](x-x_0) +  f[x_0,x_1,x_2](x-x_0)(x - x_{1}) $$
                $$P_{2}(x) = 1 + .5(x + 1) + 1(x + 1)(x - 0)$$
                $$P_{2}(x) = 1.5x + 1.5 + x^2$$
            \end{enumerate}
            \color{black}
         \end{enumerate}
        
%Problem 2
        \item (20pts, 10 each)
        \begin{enumerate}
        \color{blue} Come Back To This: WARNING!! Long\color{black}\\
            \item Given a function f defined on $[a,b]$ and set of nodes $ a =  x_0 < x_1, \cdot \cdot \cdot < x_n = b$. Define a cubic spline interpolant $S(x)$ of $f(x)$ (Hint: List the conditions $s(x)$ has to satisfy).\\
            
            \color{red}
            Consider the following formula
            $$S_{j}(x) = a_j + b_j(x - x_j) + c_j(x-x_j)^2 + d_j(x - x_j)^3$$
            We list the conditions $s(x)$ has to satisfy. From the Book.
            \begin{enumerate}
                \item $S(x)$ is a cubic polynomial, denoted $S_{j}(x)$, on the subinterval $[x_{j},x_{j+1}]$;
                \\
                We know that $a = x_0$ and $b = x_n$. Let $x_0, x_{n - 1},x_n, f(x_0) = x_0, f(x_{n-1}) = x_{n - 1}, f(x_n) = x_n $
                $$S_{0}(x) = a_{0} + b_{0}(x - x_{0}) + c_{0}(x - x_0)^2 + d_{0}(x-x_0)^3, x_0 \leq x \leq x_{n-1}$$
                $$S_{1}(x) = a_1 + b_1(x - x_{n - 1}) + c_{0}(x - x_{n - 1}) + d_{0}(x - x_{n - 1}), x_{n - 1} \leq x \leq x_n$$
                8 unknowns, $a_0,b_0,c_0,d_0, a_1,b_1,c_1,d_1$
                \\
                \item $S_{j}(x_j) = f(x_{j})$ and $S_{j}(X_{j + 1}) = f(x_{j + 1})$ for each $j = 0,1,\dots,n-1$;
                $$S_0(x_0) = f(x_0)$$
                $$S_0(x_0) = a_{0} = f(x_0)$$
                The second part
                $$S_{0}(x_{1}) = a_0 + b_0(x_1 - x_0) + c_0(x_1 - x_0)^2 + d_0(x_1 - x_0)^3 = f(x_1)$$
                \item $S_{j + 1}(x_{j + 1}) = S_{j}(x_{j + 1})$ for each $j = 0,1,\dots,n - 2$;
                $$S_{1}(x_1) = a_1 + b_1(x_1 - x_{n - 1}) + c_1(x_1 - x_{n - 1})^2 + d_1(x_1 - x_{n - 1})^3$$
                $$S_{0}(x_1) = a_0 + b_0(x_1 - x_0) + c_0(x_1 - x_0)^2 + d_0(x_1 - x_0)^3$$
                \item $S'_{j + 1}(x_{j + 1}) = S'_{j}(x_{j + 1})$ for each $j = 0,1,\dots,n-2$;
                $$S'_{1}(x_1) = b_1(x_1 - x_{n - 1}) + 2c_1(x_1 - x_{n - 1}) + 3d_1(x_1 - x_{n - 1})^2$$
                \item $S''_{j + 1}(x_{j + 1}) = S''_{j}(x_{j + 1})$ for each $j = 0,1,\dots,n-2$;
                \item  One of the following sets of boundary conditions is satisfied: 
                \begin{enumerate}
                    \item $S''(x_0) = S''(x_n) = 0$ (\textbf{natural} (or free) \textbf{boundary};
                    \item $S'(x_0) = f'(x_0)$ and $S'(x_n) = f'(x_n)$ (\textbf{clamped boundary})
                \end{enumerate}
            \end{enumerate}
            
    
            \color{black}
            \item Construct the natural cubic spline interpolant for the following data:
            \begin{center}
                \begin{tabular}{c|c}
                    x & $f(x)$ \\
                    \hline
                    0 & 1\\
                    .5 & 2.72\\
                \end{tabular}
            \end{center}
            \color{red}
            \begin{enumerate}
                \item We know that\\ $x_0 = 0, x_1 = .5$\\
                $ f(x_0) = 1,f(x)=(x_1) = 2.72$
                \item The cubic spline for the interval $[x_0,x_1]$ is as follows
                $$S_0(x) = a_0 + b_0(x - x_0) + c_0(x - x_0)^2 + d_0(x - x_0)^3$$
                $$\text{ for } x_0 = 0$$
                $$S_0(x) = a_0 + b_0(x - 0) + c_0(x - 0)^2 + d_0(x - 0)^3$$
                \item $S_{j}(x_{j}) = f(x_{j})$ and $S_{j}(x_{j + 1}) = f(x_{j + 1})$ Show it meets such conditions. 
                \begin{enumerate}
                    \item[$S_j(x_j) = f(x_j)$] $$S_0(0) = f(0)$$
                $$S_0(0) = a_0 + b_0(0 - 0) + c_0(0 - 0)^2 + d_0(0 - 0)^3 = a_0$$
                $$S_0(0) = f(0) = a_0 = 1$$
                \item[$s_j(x_{j + 1}) = f(x_{j + 1})$] Recall $a_0 = 1$ $$S_0(.5) = f(.5)$$
                $$S_0(.5) = a_0 + b_0(.5 - 0) + c_0(.5 - 0)^2 + d_0(.5 - 0)^3 $$
                $$S_{0}(.5) = 1 + (.5)b_{0} + (.25)c_0 + (.1250)d_0 = 2.72$$
                $$S_{0}(.5) = (.5)b_{0} + (.25)c_0 + (.1250)d_0 = 2.72 - 1$$
                \end{enumerate}
                \item  One of the following sets of boundary conditions is satisfied: 
                \begin{enumerate}
                    \item $S''(x_0) = S''(x_n) = 0$ (\textbf{natural} (or free) \textbf{boundary};
                    \item $S'(x_0) = f'(x_0)$ and $S'(x_n) = f'(x_n)$ (\textbf{clamped boundary})
                \end{enumerate}
                We do the following, since we where asked to find the \textbf{Natural cubic spline}
                $$S''(x_0) = S''(x_n) = 0$$
                $$S'(x) =  b_0 + 2c_0(x) + 3d_0(x)^2$$
                $$S''(x) =  2c_0 + 6d_0(x) = 0  $$
                Remember that we want $S''(x)$ to be zero, therefore $c_0 = 0$, $2c_0 + 6d_0(x) = 0$ in order for this to be true. 
                $$S''(0) = 2c_0 + 6d_0(0) = 0 \rightarrow d_0 = 0 $$
                $$S''(.5) =2c_0 + 6d_0(.5) = 0 \rightarrow c_0 = 0$$
                We have found $c_{0}, d_{0}$
                 $$S_{0}(.5) = (.5)b_{0} + (.25)c_0 + (.1250)d_0 = 2.72 - 1$$
                  $$S_{0}(.5) = b_{0} = \frac{2.72 - 1}{.5}$$
                Therefore we have found all the necessary values. 
                $$S_0(x) = 1 + 3.5(x - 0) = 1 + 3.5x$$
            \end{enumerate}
%Problem 3
            \color{black}
        \end{enumerate}
        \item (20pt, 5,15)
        \begin{enumerate}
            \item What is the order of the error bound while using an (n+ 1)-point formula approximating $f'(x_0)$?
            \color{red}
            \begin{enumerate}
                \item The erorr bound while using $n + 1$ point formula is as follows\\
                The order is $O(h^3)$
            \end{enumerate}
            \color{black}
            \item Use the most accurate 3 points formula to determine each missing entry in the table below:
            \begin{center}
                \begin{tabular}{c|c|C}
                     x & $f(x)$ & $f'(x)$ \\
                     7.1 & 1 & \color{red} 35 \color{black}\\
                     7.2 & 3 & \color{red} 5 \color{black}\\
                     7.3 & 2 & \color{red} - 25 \color{black}
                \end{tabular}
            \end{center}
            We recall some formulas
            $$f'(x_0) = \frac{f(x_0 + h) - f(x_0)}{h} - \frac{hf''(\xi)}{2}$$
            $$f'(x_0) = \frac{f(x_0 + h) - f(x_0 - h)}{2h} - \frac{h^2f'''(\xi)}{6}$$
            $$f'(x_0) = \frac{3f(x_0) - 4f(x_0 + h) + f(x_0 - 2h)}{2h} - \frac{h^2f^{(3)}(\xi)}{3}$$
            $$f'(x_0) = \frac{-3f(x_0) + 4f(x_0 + h) - f(x_0 + 2h)}{2h} - \frac{h^2f''(\xi)}{3}$$
            \color{red}
            The following formulas apply here. 
            \begin{enumerate}
                \item \color{blue}\texbf{Three-Point Endpoint Formula}\color{black}\\
                $$\bullet f'(x_0) = \frac{1}{2h}[-3f(x_0) + 4f(x_0 + h) - f(x_0 + 2h)] + \frac{h^2}{3}f^{(3)}(\xi_0)$$
                Where $\xi_0$ lies between $x_0$ and $x_0 + 2h$
                \item \color{blue}Three-Point Midpoint Formula\color{black}\\
                $$\bullet f'(x_0) = \frac{1}{2h}[f(x_0 + h) - f(x_0 - h)] - \frac{h^2}{6}f^{(3)}(\xi_1)$$
                Where $\xi_1$ lies between $x_0 - h$ and $x_0 + h$
            \end{enumerate}
            \begin{enumerate}
                \item[$x = 7.1$]
                \begin{enumerate}
                \item Consider the above formulas. We will ignore the error for the moment. 
                \item We are given 3 points, We will use more than one formula depending on the value, to calculate endpoints we can only use endpoint formulas, do not use midpoints. 
                \item Let $x_0 = 7.1, x_1 = 7.2, x_2 = 7.3$, $x_0$ is an endpoint, and we know that $h = 7.2 - 7.1 = .1$. Consider the endpoint formula.
                $$f'(x_0) = \frac{-3f(x_0) + 4f(x_0 + h) - f(x_0 + 2h)}{2h} - \frac{h^2f''(\xi)}{3}$$
                Plug in h and $x_0$
                $$f'(x_0) = \frac{-3f(7.1) + 4f(7.1 + .1) - f(7.1 + 2(.1))}{2(.1)} - \frac{(.1)^2f''(\xi)}{3}$$
                $$f'(x_0) = \frac{-3f(7.1) + 4f(7.2) - f(7.3)}{.2} - \frac{(.1)^2f''(\xi)}{3}$$
                Now consider the given values of $f(x_0), f(x_1), f(x_1)$ given by the tables, as one can see this endpoint formula has all values defined.
                $$f'(x_0) = \frac{-3(1) + 4(3) - 2}{.2} - \frac{(.1)^2f''(\xi)}{3}$$
                $$f'(x_0) = \frac{7}{.2} = 35 $$
            \end{enumerate}
            \item[$x = 7.2$]
                \begin{enumerate}
                    \item As one can see $x = 7.2$ is not an endpoint but a midpoint, using an endpoint formula would not be reasonable as we would need $f(7.4)$ which is not given. 
                    $$ f'(x_0) = \frac{1}{2h}[f(x_0 + h) - f(x_0 - h)] - \frac{h^2}{6}f^{(3)}(\xi_1)$$
                    \item Once again determine our $h = 7.3 - 7.1 = .1$, it is equally spaced so it is not surprising that h is the same as previously. 
                    \item We now compute. Ignore the error. 
                    $$ f'(x_0) = \frac{1}{2(.1)}[f(7.2 + .1) - f(7.2 - .1)] - \frac{h^2}{6}f^{(3)}(\xi_1)$$
                    $$ f'(x_0) = \frac{1}{2(.1)}[f(7.2 + .1) - f(7.2 - .1)] $$
                    $$ f'(7.2) = \frac{1}{2(.1)}[f(7.3) - f(7.1)] $$
                    $$f'(7.2) = \frac{1}{.2}[2 - 1] = 5$$
                \end{enumerate}
            \item[$x = 7.3$]
                \begin{enumerate}
                    \item $x = 7.3$ is an endpoint, we use the endpoint formula. Next $h = 7.2 - 7.3 = -.1$, take not of this. 
                    \item Next we compute. Ignore the error.
                        $$f'(x_0) = \frac{1}{2h}[-3f(x_0) + 4f(x_0 + h) - f(x_0 + 2h)] + \frac{h^2}{3}f^{(3)}(\xi_0)$$
                    $$f'(7.3) = \frac{1}{2(-.1)}[-3f(7.3) + 4f(7.3 + (-.1)) - f(7.3 + 2(-.1)] $$
                    $$f'(7.3) = \frac{1}{-.2}[-3(2) + 4(3) - (1)] $$
                    $$f'(7.3) = \frac{1}{-.2}[5] = -25 $$
                        
                \end{enumerate}
            \end{enumerate}
            \color{black}
        \end{enumerate}
%Problem 4
        \item (20 pts, 5 each)
        \begin{enumerate}
            \item Give the general expression of the Lagrange polynomials $L_{n,k}(x)$
            \color{red}
            $$L_{n,k}(x) = \prod_{i = 0;j \neq k}^{n} \frac{x - x_i}{x_{k} - x_{i}} $$
            \color{black}
            \item What is the name of the polynomial interpolant $P(x)$ of f such that $P(x_i) =f(xi),i= 0,...,n$ and $P'(x_i) =f'(x_i),i= 0,...,n$?
            \color{red}
            \\
            Hermite polynomials
            \\
            \color{black}
            \item What kind of polynomials are used for Gaussian quadratures ?\\
            \color{red}
            \\
           Legendre polynomials
            \color{black}
            \item Explain adaptive quadrature methods in no more than two lines.\\
            \color{red}
            We are given a tolerance, when the function behaves badly that is it is largely curved at many points we use more number of points in our method, when that is not the case we use less points. 
            \color{black}
        \end{enumerate}
%Problem 5
        \item (20pts, 16,4)Below is a MATLAB implementation of the trapezoid rule with some code missing.
        \begin{enumerate}
            \item 
            \color{red}
            \begin{lstlisting}
            function [I] = trapezoid_rule(f,a,b,N)
            %function to approximate $\int_a^b f(x)dx$ using the trapezoid rule
            %INPUTS %f is the function at hand
            %a is the lower bound of the interval
            %b is the upper bound of the interval
            % N is the number of panels used
            %OUTPUTS:
            % I is the approximate integral
                I = 0;
                h = (b - a)/N;
                c = a:h:b;
                for j = 0: N
                    X_j = a + j*h;
                    X_jp1 = (h/2)*(f(c(j)) + f(c(j + 1)));
                    I = I + x_jp1;
                end
            end
            \end{lstlisting}
            \color{black}
        \item Explain the differences between the Trapezoid rule and the Simpson�s rule (number of points needed,order of the error bound, etc.).
        \color{red}
        \begin{enumerate}
            \item[derivation] 
            \begin{enumerate}
                \item Trapezoid consists of the summation of trapezoid, that is there is a rectangle and a triangle, a slanted line.
                \item Simpson's rule uses a rectangle and a parabola, making it slightly more accurate. 
                \end{enumerate}
            \item[order] 
            \begin{enumerate}
                \item Rate of Trapezoid Rule: Standard Trapezoid $O(h^3)$, Composite $O(h^2)$
            \item Rate of Simpsons Rule: Standard Simpsons $O(h^5)$, Composite $O(h^4)$
            \end{enumerate}
            
            \item[error bound]
            \begin{enumerate}
                \item Composite Simpson's rule error $\frac{b - a}{180}h^4f^{4}(\mu)$
                
            \end{enumerate}
            \color{black}
        \end{enumerate}
        
\end{enumerate}
\end{enumerate}
\end{document}